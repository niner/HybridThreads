%%% Einfaches Template f�r den Praktikumsbericht (2. Bachelorarbeit)

\documentclass[praktikum,german]{hgbthesis}
\graphicspath{{images/}}   				% wo liegen die EPS-Bilder?
%\AddBibFile{literatur.bib}  % Angabe der BibTeX-Datei


%%%----------------------------------------------------------
\begin{document}
%%%----------------------------------------------------------

\author{Stefan Seifert}
\studiengang{Software Engineering}
\studienort{Hagenberg}
\abgabemonat{Februar}
\abgabejahr{2012}

\nummer{1010307037-B}
\betreuer{Mag. Thomas Seel}
\firma{%
   Atikon EDV \& Marketing GmbH\\
   4020 Leonding, Kornstra�e 4
}
\firmenTel{+43 732 611266}
\firmenUrl{www.atikon.com}

%%%----------------------------------------------------------
\frontmatter
\maketitle
\tableofcontents
%%%----------------------------------------------------------

\chapter{Kurzfassung}
Umfang der Kurzfassung: ca.\ 200 Worte.

Zum allgemeinen Inhalt des Berichts: Dieser Bericht beschreibt den Ablauf des Praktikums, die Aufgaben und durchgef�hrten Projekte und Erfahrungen. Die eigenen Aktivit�ten (Projekte) stehen dabei nat�rlich im Mittelpunkt und bilden den Hauptteil des Berichts. Wenn viele Kleinprojekte bearbeitet wurden, sollten einige davon exemplarisch
genauer beschrieben werden. Neben der eigentlichen Arbeit sollten aber auch folgende weitere Aspekte ber�cksichtigt werden:
%
\begin{itemize}
\item Abl�ufe (Workflows) innerhalb des Unternehmens bzw. in Projekten (grafische Darstellungen
k�nnen dabei n�tzlich sein)
\item Arbeits- und F�hrungsstil, Kommunikation innerhalb des Unternehmens
\item Kommunikation nach au�en (Kunden, Partner)
\item Zeitsituation, Terminprobleme
\item Einbettung in das Team, soziale Erfahrungen
\item Einsatz von speziellen Techniken, Methoden und Werkzeugen.
\item Wichtige Herausforderungen oder Schwierigkeiten
\item Anforderungen in Bezug auf die Ausbildung im Studium (gut einsetzbare Kenntnisse, Defizite)
\end{itemize}
%
Die nachfolgenden Kapitel�berschriften sollen nur zur Orientierung f�r die Struktur des Berichts dienen, �ber die konkrete Einteilung und den Wortlaut kann man nat�rlich selbst entscheiden.

%%%----------------------------------------------------------
\mainmatter           %Hauptteil (ab hier arab. Seitenzahlen)
%%%----------------------------------------------------------

\chapter{Das Unternehmen}
Atikon EDV \& Marketing GmbH mit Sitz in Leonding bietet in den 11 Jahren ihres Bestehens mit 55 Mitarbeitern Marketingdienste f�r Steuerberater und �rzte und entwickelt Beratungssoftware f�r Steuerberater. Diese Spezialisierung spiegelt sich in der Kundenstruktur wieder. Unter den ca. 3400 Kunden befinden sich in etwa 80 \% Steuerberater, 10 \% �rzte und 10 \% sonstige Unternehmer.

Diesen Kunden werden im Marketingbereich Websites mit individuellem Design und verschiedenste Drucksorten angeboten. Dazwischen gibt es noch Hybridprodukte wie Steuernews und Steuerinfo, welche in die Webseite eingebunden oder gedruckt verteilt werden k�nnen. Abgerundet wird das Portfolio durch Onlinetools wie Steuerrechner, z.B. zur Brutto/Netto Berechnung oder zur Einkommensteuerberechnung und einem Newslettersystem.

Die Beratungssoftware wird vom Team der Windowsprogrammierung erstellt. Hier gibt es eigenvertriebene Produkte wie die "Formel-f" oder Software zur Bilanzpr�sentation und Auftragsarbeiten, z.B. f�r die deutsche DATEV.

Das f�nfk�pfige Team der Webprogrammierung k�mmert sich um Webanwendungen sowohl f�r den internen Gebrauch, als auch f�r externe Kunden. Dazu geh�ren:
%
\begin{itemize}
\item das Content Management System
\item Projektverwaltung
\item Onlinetools (z.B. die Steuerrechner)
\item Newslettersystem
\item Kundenprojekte wie z.B. ein Online-CRM-System
\end{itemize}

\chapter{Projekte und T�tigkeiten w�hrend des Praktikums}

\section{CMS3000}
Umfang: 2--3 Seiten (Projektziel(e), Projektumfeld)

Websites sind eines der Hauptprodukte der Firma. Die Kundenstruktur findet sich auch abgebildet auf die ca. 1600 bisher erstellten Websites wieder. Steuerberater sind zumeist eher kleine Unternehmen. Es gibt jedoch unter ihnen auch Netzwerke aus zahlreichen Einzelunternehmen mit zusammen 100en Standorten und 1000en Mitarbeitern. Dementsprechend sind auch bis auf einzelne Ausnahmen die meisten der von uns erstellten und betreuten Websites relativ klein. Die wichtigsten Verkaufsargumente sind die einerseits verf�gbaren branchenspezifischen Inhalte wie z.B. regelm��ig aktualisierte Steuernews, andererseits aber die vollst�ndige Individualisierung von Layouts und Websitestruktur.

Dementsprechend gestalten sich daher auch die Anforderungen an das Content Management System:
%
\begin{itemize}
\item Unterst�tzung f�r 1000e individuelle Websites in einem System
\item Einfache Verteilung und Updates von gemeinsam genutzten Inhalten
\item V�llige Freiheit bei Design und Struktur
\item Individuelle Erweiterungen
\item Einfache Bedienung f�r Kunden mit Eigenwartung
\end{itemize}

In den ersten 10 Jahren des Firmenbestehens hat das deutsche Produkt "ZMS" die Anforderungen am besten erf�llt. Es handelt sich dabei um freie, GPL lizenzierte Software, die in Python geschrieben ist und auf dem Zope Application Server basiert.

\chapter{Projektbeispiele}
Umfang: 5--6 Seiten (Umsetzung, grober Terminplan, Ergebnisse, Qualit�tssicherungsma�nahmen)
   
   
\chapter{Erfahrungen und Zusammenfassung}
Umfang: 1--2 Seiten


Zum allgemeinen Inhalt des Berichts: Dieser Bericht beschreibt den Ablauf des Praktikums, die Aufgaben und durchgef�hrten Projekte und Erfahrungen. Die eigenen Aktivit�ten (Projekte) stehen dabei nat�rlich im Mittelpunkt und bilden den Hauptteil des Berichts. Wenn viele Kleinprojekte bearbeitet wurden, sollten einige davon exemplarisch und im Detail beschrieben werden.


%%%----------------------------------------------------------
% Quellenverzeichnis (sofern notwendig, sonst weglassen)
%\MakeBibliography{Quellenverzeichnis}

%%%Messbox zur Druckkontrolle
%\input{messbox.tex}

\end{document}

% vim: spell spelllang=de_at
